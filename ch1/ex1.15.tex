\documentclass[12pt,a4paper]{article}
\usepackage{amsmath}
\begin{document}
\noindent \textbf{Exercise 1.15.} The sine of an angle (specified in radians) can be computed by making use of the approximation $\sin x \approx  x$ if x is sufficiently small, and the trigonometric identity\\\\
$\sin x = 3\sin {\dfrac{x}{3}}-4{\sin ^3 {\dfrac{x}{3}}}$\\\\
to reduce the size of the argument of $\sin$. (For purposes of this exercise an angle is considered ``sufficiently small'' if its magnitude is not greater than 0.1 radians.) These ideas are incorporated in the following procedures:
\begin{verbatim}
(define (cube x) (* x x x))
(define (p x) (- (* 3 x) (* 4 (cube x))))
   (if (not (> (abs angle) 0.1)
      angle
      (p (sine (/ angle 3.0)))))
\end{verbatim}
a. How many times is the procedure \verb|p| applied when \verb|(sine 12.15)| is evaluated?\\\\
b. What is the order of growth in space and number of steps (as a function of $a$) used by the process generated by the \verb|sine| procedure when \verb|(sine a)| is evaluated?\\\\
\textbf{Answer}\\
a. 5 times.\\\\
b. The order of growth of the number of steps is $\Theta (\log a)$. Since the procedure is a recursive procedure, the order of growth in space grows linearly with the number of the recursive steps. So the order of growth in space is also $\Theta(\log a)$.
\end{document}
